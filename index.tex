\documentclass[12pt,a4paper]{article}
\usepackage{graphicx}
% For Verbatim environment and fvset:
\usepackage{fancyvrb}
% For reduced margin between Verbatims and captions
%\usepackage[skip=3mm]{caption}
\setlength{\abovecaptionskip}{-4mm}
\setlength{\belowcaptionskip}{0mm}

\fvset{frame=single, framesep=2mm, numbers=left, numbersep=2mm, tabsize=0}
\author{Francisco Castro}
\title{Advanced Sed}

%%%%%%%%%%%%%%%%%%%%%%%%%%%%%%%%%%%%%%%%%%%%%%%%%%%%%%%%%%%%%%%%%%%%%%%%%%%%%
% Support for TODO generation:
% \todo{Thing that has to be done.}
%     - Adds a box at the current location showing the text inside a rectangle
% \todos
%     - Prints a list of the \todo items in the document referencing the page
%       numbers where they were mentioned.
%     - Note: Must be used after the last \todo.

\def\todounexpandeditems{}
\def\todoitem{\noexpand\todoitem}
\def\todo#1{
  \fbox{{\textbf TODO:} #1}
  \edef\todounexpandeditems{\todounexpandeditems\todoitem Pg \thepage: #1}}
\def\todos{
  \begin{itemize}
    \def\todoitem{\item}
    \todounexpandeditems
  \end{itemize}}

%%%%%%%%%%%%%%%%%%%%%%%%%%%%%%%%%%%%%%%%%%%%%%%%%%%%%%%%%%%%%%%%%%%%%%%%%%%%%

% Fibonacci function in the same style as \max or \lim:
\newcommand{\fib}{\mathop{\mathrm{fib}}}

% The word `sed' in monospace font
\newcommand{\sed}{{\tt sed} }


\begin{document}
\maketitle

\section{Table of Contents}
\tableofcontents

%%%%%%%%%%%%%%%%%%%%%%%%%%%%%%%%%%%%%%%%%%%%%%%%%%%%%%%%%%%%%%%%%%%%%%%%%%%%%

\section{Introduction}

I started writing this text in September 2021, several decades well after \sed*
was invented, yet even though there were a few books among other resources
providing introductions and lists of use cases where \sed* is still a relatively
good idea in the context of a simple shell script.  Here we have another goal,
to explore this tool as an esoteric programming language, trying to exploit it
in order to achive uses way outside its imagined scope when it was created.

\todo{Write about resources to start with sed, including the standard}

\todo{Write about sed memory, maybe in another section?}

%%%%%%%%%%%%%%%%%%%%%%%%%%%%%%%%%%%%%%%%%%%%%%%%%%%%%%%%%%%%%%%%%%%%%%%%%%%%%

\section{Languages}

In this section we will provide the set of definitions from Language Theory
used in this text.  This is only done in order to have a common language.
\subsection{Regular Expressions}
\subsection{Parsing Expression Grammar}

%%%%%%%%%%%%%%%%%%%%%%%%%%%%%%%%%%%%%%%%%%%%%%%%%%%%%%%%%%%%%%%%%%%%%%%%%%%%%

%%%%%%%%%%%%%%%%%%%%%%%%%%%%%%%%%%%%%%%%%%%%%%%%%%%%%%%%%%%%%%%%%%%%%%%%%%%%%
\section{Toolset}
%%%%%%%%%%%%%%%%%%%%%%%%%%%%%%%%%%%%%%%%%%%%%%%%%%%%%%%%%%%%%%%%%%%%%%%%%%%%%

%%%%%%%%%%%%%%%%%%%%%%%%%%%%%%%%%%%%%%%%%%%%%%%%%%%%%%%%%%%%%%%%%%%%%%%%%%%%%
\subsection{Unary Math}

\begin{description}
	\item[Addition: $x^ny^m\mapsto x^{n+m}$:]
		\begin{verbatim}
			s/y/x/g
		\end{verbatim}

	\item[Duplication: $x^n\mapsto x^{2n}$:]
		\begin{verbatim} s/xx*/&&/ \end{verbatim}

	\item[Assignment: $x^n\mapsto x^ny^m$:]

		This can be achieved by an intermediate representation containing a pipe in
		the middle of a duplication: $x^n\texttt{|}x^n$, that way each of the
		{\tt x}s after the pipe can be converted one by one into {\tt y}s.  At the
		i-th iteration we will have:
		$x^ny^{i-1}\texttt{|}xx^{n-i} \mapsto x^ny^i\texttt{|}x^{n-i}$.

		\begin{Verbatim}
			s/x*/&|&/; :a; s/|x/y|/; ta; s/|//
		\end{Verbatim}

	\item[Multiplication: $x^ny^m\mapsto x^{n(m+1)}$:]

		Since $n\cdot m=n(m-1)+n$, this can be solved by consuming one {\tt y}
		at a time by repeating the substitution $x^ny^m \equiv x^nyy^{m-1}
		\mapsto	x^ny^{m-1}x^n$ while there are {\tt y} availables.  The last
		substitution will then be $x^nyx^{n(m-1)} \mapsto x^nx^nx^{n(m-1)}
		\equiv x^{n(m+1)}$:

		\begin{Verbatim}
			:a; s/\(x*\)y\(y*\)/\1\2\1/; ta
		\end{Verbatim}

		To calculate $n\cdot m$ instead of $n(m+1)$, we can either delete one
		{\tt y} if we know there is at least one (\verb|s/y//|); or we can place
		a separator between the {\tt x}s and the {\tt y}s, then at the iteration
		we will have $x^n\texttt{|yx}^{n(m-1)} \mapsto x^n\texttt{|x}^{nm}$:

		\begin{Verbatim}
			s/x*/&|/
			:a; s/\(x*\)|y\(y*\)/\1|\2\1/; ta
			s/x*|//
		\end{Verbatim}

	\item[Factorization: $\texttt{x}^n \mapsto
		\texttt{|x}^{p_1}\texttt{|x}^{p_2}\texttt{|}
		\ldots\texttt{|x}^{p_k}\texttt{|}$:]

		We can use the following code to compute the prime numbers
		$p_1, p_2\ldots p_n$, where $n=\prod_{i=1}^k p_i$.
		Additionally this prime factors will be sorted
		decreasingly, assuring a unique decomposition.

		\begin{Verbatim}
			s/x*/|&|/; :a
			  s/|\(xxx*\)\(\1\1*\)|/|\1<x>\2|/
			  s/\(x*\)<\(x*\)>\1/\1<\2x>/
			  s/<\(x*\)>|/|\1|/
			ta
		\end{Verbatim}

		\begin{enumerate}
			\item On the first line we wrap the {\tt x}s between pipes just to
				make processing simple, it is not required though.

			\item Then for a block of length $n$ this is when we find a 
				divisor $d>1$ of $n$, the backreference is placed twice to ensure
				$d<n$.  Since the algorithm for matching regular expressions is
				greedy, \verb|\1| will match the largest divisor smaller than $n$,
				meaning that $n/d$ is the smallest prime that divides $n$.

				The block $\texttt{|x}^n\texttt{|}$ will be converted into
				$\texttt{|x}^d\texttt{<x>x}^{n-d}\texttt{|}$.

			\item The next substitution gradually substracts $d$ from the
				``number'' after the \verb|>| symbol, adding one before it.
				In other words we just divided $n$ over $d$ using repeated
				substraction.
				So after applying this rule as many times as possible, we obtain:
				$\texttt{|x}^d\texttt{<x}^{n/d}\texttt{>|}$.

			\item And the final rule, applicable only when the division has
				completed, is to convert it back to the ``pipe separated
				unary format'' so that we can proceed to find newer factors:
				$ \texttt{|x}^n\texttt{|} \mapsto \ldots
				\mapsto \texttt{|x}^d\texttt{|x}^{n/d}\texttt{|}$
		\end{enumerate}

	\item[Logarithm: $x^n\mapsto y^{\lfloor\log_2 n\rfloor}$:]

		The idea here is to divide by 2 repeatedly until {\tt xx} can no longer
		be found.  Note that we divide by 2 rounding down by substituting a
		single (final) {\tt x} in the pattern at line 2 to an empty string.
		There may be a remaining {\tt x} which we remove at the end.

		\begin{Verbatim}
			:a; /xx/ {
			  s/x\(x\|\)/\1/g
			  s/^/y/
			ba }
			s/x//
		\end{Verbatim}

	\item[Minimum: $\texttt{x}^n\texttt{|x}^m \mapsto
		\texttt{x}^{\min\{n,m\}}$:]

		% sedcode minimum:
		Calculating the minimum between $n$ and $m$ is performed by
		substituting them leaving only the longest common prefix.  Such a
		transformation is simply \verb!s/\(x*\)x*|\1x*/\1/!.

	\item[Maximum: $\texttt{x}^n\texttt{|x}^m \mapsto
		\texttt{x}^{\max\{n,m\}}$:]

		To fully comprehend how the implementation presented here
		we need to embrace the fact that \sed uses a greedy algorithm for
		substitutions. In other words whenever our regular
		expression is the concatenation of two or more quantified parts, it
		will try to match as much as possible in the first part, then as
		much as possible in the second and so on, as long the regex matches.

		When looking at the implementation for minimum we can see that the
		only difference is that we are capturing and keeping the second and
		third occurrences of $\texttt{x}^*$:

		% sedcode maximum:
		\begin{Verbatim}
			s/\(x*\)\(x*\)|\1\(x*\)/\1\2\3/
		\end{Verbatim}

		So, since
		\begin{itemize}
			\item \verb|\1| captures $\min\{n,m\}$ number of \verb|x|s,
			\item \verb|\2| captures $n-\min\{n,m\}$ number of \verb|x|s, and
			\item \verb|\3| captures $m-\min\{n,m\}$ number of \verb|x|s;
		\end{itemize}
		then for \verb|\1\2\3|: $n+m-\min\{n,m\}=\max\{n,m\}$.

		Another possibility, though way less elegant, is to abuse the fact
		that we can assume the greedy behavior for the alternative operator
		``\verb!\|!'' as well.  If $n<=m$ then \verb|\1\3| will match $n +
		(m-n)=m$, and if $n>m$, then the second alternative will be selected
		leaving \verb|\3| empty:
		\begin{Verbatim}
			s/\(x*\)|\(\1\(x*\)\|x*\)/\1\3/
		\end{Verbatim}

	\item[Fibonacci: $\texttt{x}^n\mapsto \texttt{x}^{\fib n}$:]

		Fibonacci is usually defined recursively by $\fib n = \fib (n-1) +
		\fib (n-2)$ with the base case $\fib n = n$ when $n<2$.  So in the
		implementation presented here we first handle the base case first by
		not adding pipes (which disables all substitutions on the following
		lines).

		The recursive case is handling by having our pattern space with the
		format $\texttt{x}^{n-i}\texttt{|x}^{\fib i-1}\texttt{|x}^{\fib i}$,
		and applying the inductive substitution $(a,b)\mapsto (b,a+b)$
		which increases by 1 the value at $i$:
		$\texttt{x}^{n-i}\texttt{|x}^{\fib i-1}\texttt{|x}^{\fib i}$
		$\mapsto$
		$\texttt{x}^{n-i-1}\texttt{|x}^{\fib i}\texttt{|x}^{\fib i+1}$.

		When $i$ reaches $n$, that is when there are no more {\tt x}s before
		the first pipe, we will have $\texttt{|x}^{\fib n-1}\texttt{|x}^{\fib n}$,
		so at that point we just leave the text after the last pipe:

		% sedcode fibonacci:
		\begin{Verbatim}
			/xx/ s/x$/||x/
			:a; s/x|\(x*\)|\(x*\)/|\2|\1\2/; ta
			s/.*|//
		\end{Verbatim}

\end{description}

%%%%%%%%%%%%%%%%%%%%%%%%%%%%%%%%%%%%%%%%%%%%%%%%%%%%%%%%%%%%%%%%%%%%%%%%%%%%%
\subsection{Binary Math}

\begin{description}
	\item[XOR: $a_n\ldots a_1 a_0\texttt{|}b_m\ldots b_1 b_0 \mapsto
		c_{\max\{n, m\}}\ldots c_1 c_0$ where $(c_i)=(a_i)\oplus(b_i)$:]$\,$

		Bit-wise XOR between $(a_i)$ and $(b_i)$, assuming as many leading zeros
		as needed.

		% sedcode bitwise-xor:
		\begin{Verbatim}
			s/$/|/
			:a; s/^|\(.*\)|/\1/; s/||//
			  s/0|\(.*\)\(.\)|/|\1|\2/
			  s/1|\(.*\)0|/|\1|1/
			  s/1|\(.*\)1|/|\1|0/; ta
			s/^0*//; s/^$/0/
		\end{Verbatim}

		\begin{enumerate}
			\item We start by adding a pipe to the end.
				We will store the result after that pipe.
			\item At the start of each iteration denoted by the label \verb|:a| we
				detect the cases where there are no more binary digits left in any
				of the operands.

				In the following 3 lines we will be processing
				one bit at a time starting from the right (least significant bit).
			\item The rule that we apply here $0\oplus x = x$.  We just maintain the
				least significant bit of the second number.
			\item Then we consider $1\oplus 0=1$.
			\item And the other case $1\oplus 1=0$.
			\item Finally we just remove leading zeroes and place a zero in place of
				an empty string.
		\end{enumerate}
\end{description}

\todo{Write about binary addition.}

\todo{Write about truncation.}

%%%%%%%%%%%%%%%%%%%%%%%%%%%%%%%%%%%%%%%%%%%%%%%%%%%%%%%%%%%%%%%%%%%%%%%%%%%%%
\subsection{Lookup Tables}

Many of the most complex programs written in \sed use the power of
backreferences in order to do translations which otherwise could be very
complex, here we will focus on two very practical use cases, converting from
decimal numbers to unary and viceversa.

\subsubsection{Decimal to Unary}

Let's say we want to convert the digits of the number {\tt 42} to unary:
{\tt |xxxx|xx}, we can use the following short program composed of just three
lines:

\begin{Verbatim}
	s/./|0123456789&!/g
	s/\(.\)[0-9]*\1//g
	:a; s/|!/|/g; s/.!/!x/g; ta
\end{Verbatim}

\begin{enumerate}
	\item with the first substitution we would obtain the seemingly useless
		string: {\tt |01234567894!|01234567892!}, yet there is an interesting
		pattern going on.
	\item If for each block we were to remove the duplicated numbers and
		everything in between, we would obtain {\tt |0123!|01!}, and that's
		what the second line does.

		Note that the number of digits left matches the corresponding digit
		that we had before.  That's because we inserted all the digits in order.

	\item Finally to obtain the {\tt x}s, we can iterate the exclamation marks
		to the left substituting the digits by {\tt x}s until they reach the
		pipes (when we remove them).
\end{enumerate}

\subsubsection{Unary to Decimal}
\todo{Copiar desde la cuadernola la explicaci\'on de las reverse lookup tables.}

\subsubsection{Generic Substitutions}
\todo{Escribir c\'omo se podr\'{\i}a usar una lookup table en general para
hacer una substitución arbitraria.}
% the example usually seen for "sed lookup tables" is that of substituting
% numbers via 0zero1one2two3three4four5five6six7seven8eight9nine, there has
% to be another simple example.
%   Maybe converting from hexa to binary requiring only two commands.

%%%%%%%%%%%%%%%%%%%%%%%%%%%%%%%%%%%%%%%%%%%%%%%%%%%%%%%%%%%%%%%%%%%%%%%%%%%%%
\subsection{Debugging}

When developing programs in \sed, more often than not we end up with mountains
of completely unreadable line noise which at a certain point, sometimes even
way before to be completed, leaves us in a state of absolute confusion.  Is at
this point that we really need to understand what's going on.

One of the simplest ways to try to understand the code, is to print the content
of the pattern space, in the most simple form we just place the {\tt p} command
whenever we want to, however if we want to print it at several places and
specially if there are newlines inside the pattern space, we might want to use
something like this instead:

\begin{Verbatim}
	... lots of code ...
	s/.*/At xyz: <<&>>/p;  s/[^<]*<<//;  s/>>$//
	... lots of code ...
\end{Verbatim}

Note that by using the previous code snippet we ``set to true'' the state of
whether a substitution was applied at that point.  If we don't care about its
state then we might just use that code, if not we have to maintain it by
basically duplicating our code around an ``if then else'':

\begin{Verbatim}
	... lots of code ...
	tTxyz; s/.*/At xyz (F): <<&>>/p; s/[^<]*<<//; s/>>$//; tFxyz
	:Txyz; s/.*/At xyz (T): <<&>>/p; s/[^<]*<<//; s/>>$//; :Fxyz
	... lots of code ...
\end{Verbatim}

This works because if the flag was true, it will jump to the {\tt (T)} part,
with these new substitutions restoring the flag to true.
Oterwise the conditional jump to {\tt Fxyz} will reset it to false
before continuing.

If the hold space has to be printed as well, then one simple solution is
logging the pattern space, followed by the {\tt x} command, that logging
the hold space and finally switching back the spaces with the {\tt x}
command once again.

\todo{add more tools}

%%%%%%%%%%%%%%%%%%%%%%%%%%%%%%%%%%%%%%%%%%%%%%%%%%%%%%%%%%%%%%%%%%%%%%%%%%%%%
\subsection{Jump Tables}

\todo{complete jump tables}

%%%%%%%%%%%%%%%%%%%%%%%%%%%%%%%%%%%%%%%%%%%%%%%%%%%%%%%%%%%%%%%%%%%%%%%%%%%%%
\section{Advanced Examples}
%%%%%%%%%%%%%%%%%%%%%%%%%%%%%%%%%%%%%%%%%%%%%%%%%%%%%%%%%%%%%%%%%%%%%%%%%%%%%

%%%%%%%%%%%%%%%%%%%%%%%%%%%%%%%%%%%%%%%%%%%%%%%%%%%%%%%%%%%%%%%%%%%%%%%%%%%%%
\subsection{uniq}

It's not really clear whether this example should be in the Advanced Examples
section, as it does not introduce nor require any non-obvious programming
patterns.

The {\tt uniq} program reads one line at a time from the standard input.
If it is equal to the previous line then it is skipped, otherwise it is printed
to the standard output.  Its pseudocode becomes extremely simple in a higher
level programming language like lua:

%\begin{figure}[h]
%	\label{code:luauniq}
	\begin{Verbatim}
	for line in io.lines() do
	  if line ~= previous_line then
	    print(line)
	    previous_line = line
	  end
	end
	\end{Verbatim}
%	\caption{Pseudocode for {\tt uniq} in lua 5.}
%\end{figure}

While the \sed* version requires us to introduce one of the least common
commands, `{\tt N}':

% sedcode uniq:
\begin{Verbatim}
p; :a; $d; N
  s/^\(.*\)\n\1$/\1/; ta
  s/.*\n//p; ba
\end{Verbatim}

Most of the time our sed programs usually contain a single `{\tt s}' command
and that's it; in those cases sed consumes one line at a time and the pattern
space, the main memory of the program, never contains newline characters:
`\verb|\n|'.  This program however is the exception.

Here our pattern space is used in two ways: It may either contain the text of a
single line, for example when the program begins; or it may as well have ``two
lines'' inside, precisely after the {\tt N} command.
`{\tt N}' reads the next line from the standard input and appends it to the
pattern space with a newline in the middle.

For instance if the input is composed of two lines: `{\tt a}' and `{\tt b}',
processing them with \verb|N;s/\n/-/| will result in `{\tt a-b}'.

Having this behavior is what allows us to write programs in which the lines
cannot be processed independently of one another.

%%%%%%%%%%%%%%%%%%%%%%%%%%%%%%%%%%%%%%%%%%%%%%%%%%%%%%%%%%%%%%%%%%%%%%%%%%%%%
\subsection{Vigen\`ere}

The Vigen\`ere cipher is one of the basic methods shown in most introductory
texts about classic cryptography.  It is based on the Caesar cipher in which the
encryption is done by ``rotating'' or ``shifting'' the characters a given
number of times.

For instance if we rotate the characters by 1 position in a Caesar cipher,
{\tt AMAZING} becomes {\tt BNBAJOH} (each character is substituted by the
next character in the alphabet).  If we use 2 positions, as expected we would
be substituting {\tt A} to {\tt C}, {\tt B} to {\tt D} and so on:
{\tt COCBKPI} for the example above.

Such a simple substitution could be trivially performed in \sed* with the
{\tt y} command.

With Vigen\'ere however, the number of rotations varies for each character and
it is defined by a ``key'' text.  We start the encryption process by repeating
the key until its length is greater or equal to the clear text.  Then each
character of the clear text will have a corresponding character of the key.
If that of the key is an {\tt A} then no rotation is performed,
for {\tt B} we rotate 1 position, for {\tt C} 2 positions in the alphabet,
3 for {\tt D}, and so on.  This means that the {\tt y} command is now out of
the question.

We now present an implementation where the key is sent as the first line,
followed by the text to encrypt:

% sedcode vigenere:
\begin{Verbatim}
	1{ h; d; }
	G; s/\(^\)\|\n/&<ABCDEFGHIJKLMNOPQRSTUVWXYZA>/g
	:a
	  s/\(<[A-Z]*>\)\([^A-Z]\)\(.*\n\)/\2\1\3/
	  s/\(<.*\(.\)\(.\).*>\)\2\(.*<.*\(.\)\(.\).*>\)\6/\1\3\4\5/
	  /\n<.*>A/  s/\(<.*>\)\(.\)\(.*>\)./\2\1\3/
	  />$/{ G; s/>\n/>/; }
	  ta
	s/<.*//
\end{Verbatim}

The state of this program is composed of the hold space which is used for
storing the encryption key, and the pattern space of the form: \emph{encrypted
text} \verb|<ABC...XYZA>| \emph{remaining plain text} \verb|\n|
\verb|<ABC...XYZA>| \emph{remaining key}.

\begin{enumerate}
	\item The key is stored into the hold space, and we proceed to the next
		line without any output: {\tt d}.
	\item We begin by appending to the clear text (in the pattern space) a
		newline and the key that we have stored in the hold space.  Before each
		of the lines of our pattern space we append the {\tt <ABC...XYZA>} text,
		which we will later use for the rotations.

		Note that on some implementations the precedence order of the carret
		and the ``or'' operator (\verb:\|:) may be different.  That why we
		enclosed the caret into a group.
	\addtocounter{enumi}{1}
	\item Here we add a rule for those characters that are not in the A-Z
		class, like punctuation marks.  ``Accepting'' a character is done
		by moving it from the beginning of the \emph{remaining plain text} to
		the end of the \emph{encrypted text}.
    \item This is the rule for the rotation, the core of this implementation.
\end{enumerate}

\subsection{Height of a Tree}

In this simple example we receive an n-ary tree and return its height in
unary.  An example input could be \verb|[1,[[5],8],10,[3,2],5]| with the
corresponding output \verb|xxx|, as it shows 3 nesting levels.

% sedcode treeheight:
\begin{Verbatim}
	s/[^],[]//g; s/,,*/,/g; s/^,*//; s/,*$//; s/\[,*/[/g; s/,*\]/]/g
	:a
	  s/\[\(x*\)\]/\1x/g
	  s/\(xx*\)\(x*\),\1\(x*\)/\1\2\3/g
	  ta
\end{Verbatim}

The first line just deletes everything but the brackets and commas, and
it also deletes any redundant comma, de-duplicating and leaving only
those that separate children, ie: {\tt [],[]}.

Then in the line 3 we handle any node with 0 or 1 child, by converting
them to their height: one more than that of their child.  Final nodes will
have height 1.

Finally on the line 4 we calculate the maximum of two consecutive heights
by substituting them to their maximum.  That way on each substitution we
will be removing one comma, eventually reaching the case when the line 3
is applicable on that node.

Since the tree representation is a finite string, it means that there can
only be a finite number of children.  Note that since we remove any nodes
but ``arrays'', all the final nodes will always be empty arrays, and so
it could be defined inductively as follows:
either a node is an empty array $\texttt{[]}\in N$, or if it is
composed of children $c_1, \ldots, c_k\in N$ that are also nodes, then
$\texttt{[}c_1\texttt{,}\ldots\texttt{,}c_k\texttt{]} \in N$.  This is why
it eventually ends.

\subsection{Dijkstra}
% Dijkstra:
We have the following format for the nodes depending on their state:
\begin{center}
	\begin{tabular}{|l|l|}
		\hline
		$\texttt{<}node\texttt{,x}^*\texttt{>}$ & Nodes to iterate.\\
		\hline
		$\texttt{<}node\texttt{;x}^*\texttt{>}$
			& New nodes created at the current iteration.\\
		\hline
		$\texttt{<}node\texttt{!x}^*\texttt{>}$ & Node being iterated.\\
		\hline
		$\texttt{<}node\texttt{.x}^*\texttt{>}$ & Iterated nodes.\\
		\hline
	\end{tabular}
\end{center}

While for the edges:
\begin{center}
	\begin{tabular}{|l|l|}
		\hline
		$\texttt{<}sourceNode\texttt{,}targetNode\texttt{,x}^*\texttt{>}$
			& Edges ready to visit.\\
		\hline
		$\texttt{<}sourceNode\texttt{,}targetNode\texttt{:x}^*\texttt{>}$
			& Visited edges.\\
		\hline
	\end{tabular}
\end{center}

% sedcode dijkstra:
\begin{Verbatim}
	:i; /<[^>,]*,x*>/!{
	  :m
	  s/\(<[^;>]*\;x*>\)\(<[^.>]*\.x*>\)/\2\1/
	  s/<\([^.;>]*\)[.;]\(x*\)xx*>\(.*<\1;\2>\)/\3/
	  s/\(<\([^,;>]*\)[;.]\(x*\)>.*\)<\2;\3x*>/\1/; tm
	  /;/!q; y/;/,/
	bi;}
	s/,/!/
	:a
	  s/\(<\([^>]*\)!\(x*\)>.*<\2,\([^,]*\)\),\(x*>\)/<\4;\3\5\1:\5/
	  ta
	y/:!/,./; bi
\end{Verbatim}

\todo{explain Dijkstra's algorithm implementation.}

\todo{add more examples}

% another advanced example could be a program to select child elements from
% json strings, the first line could be the selector like "foo".0."id", and the
% remaining lines could be json objects (which could span several lines
% themselves).


%%%%%%%%%%%%%%%%%%%%%%%%%%%%%%%%%%%%%%%%%%%%%%%%%%%%%%%%%%%%%%%%%%%%%%%%%%%%%

\section{Appendices}
\appendix

\section{TODOs}
\todos

\end{document}
