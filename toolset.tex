%%%%%%%%%%%%%%%%%%%%%%%%%%%%%%%%%%%%%%%%%%%%%%%%%%%%%%%%%%%%%%%%%%%%%%%%%%%%%
\section{Toolset}
%%%%%%%%%%%%%%%%%%%%%%%%%%%%%%%%%%%%%%%%%%%%%%%%%%%%%%%%%%%%%%%%%%%%%%%%%%%%%

%%%%%%%%%%%%%%%%%%%%%%%%%%%%%%%%%%%%%%%%%%%%%%%%%%%%%%%%%%%%%%%%%%%%%%%%%%%%%
\subsection{Unary Math}

\begin{description}
	\item[Addition: $x^ny^m\mapsto x^{n+m}$:]
		\begin{verbatim}
			s/y/x/g
		\end{verbatim}

	\item[Duplication: $x^n\mapsto x^{2n}$:]
		\begin{verbatim} s/xx*/&&/ \end{verbatim}

	\item[Assignment: $x^n\mapsto x^ny^m$:]

		This can be achieved by an intermediate representation containing a pipe in
		the middle of a duplication: $x^n\texttt{|}x^n$, that way each of the
		{\tt x}s after the pipe can be converted one by one into {\tt y}s.  At the
		i-th iteration we will have:
		$x^ny^{i-1}\texttt{|}xx^{n-i} \mapsto x^ny^i\texttt{|}x^{n-i}$.

		\begin{Verbatim}
			s/x*/&|&/; :a; s/|x/y|/; ta; s/|//
		\end{Verbatim}

	\item[Multiplication: $x^ny^m\mapsto x^{n(m+1)}$:]

		Since $n\cdot m=n(m-1)+n$, this can be solved by consuming one {\tt y}
		at a time by repeating the substitution $x^ny^m \equiv x^nyy^{m-1}
		\mapsto	x^ny^{m-1}x^n$ while there are {\tt y} availables.  The last
		substitution will then be $x^nyx^{n(m-1)} \mapsto x^nx^nx^{n(m-1)}
		\equiv x^{n(m+1)}$:

		\begin{Verbatim}
			:a; s/\(x*\)y\(y*\)/\1\2\1/; ta
		\end{Verbatim}

		To calculate $n\cdot m$ instead of $n(m+1)$, we can either delete one
		{\tt y} if we know there is at least one (\verb|s/y//|); or we can place
		a separator between the {\tt x}s and the {\tt y}s, then at the iteration
		we will have $x^n\texttt{|yx}^{n(m-1)} \mapsto x^n\texttt{|x}^{nm}$:

		\begin{Verbatim}
			s/x*/&|/
			:a; s/\(x*\)|y\(y*\)/\1|\2\1/; ta
			s/x*|//
		\end{Verbatim}

	\item[Factorization: $\texttt{x}^n \mapsto
		\texttt{|x}^{p_1}\texttt{|x}^{p_2}\texttt{|}
		\ldots\texttt{|x}^{p_k}\texttt{|}$:]

		We can use the following code to compute the prime numbers
		$p_1, p_2\ldots p_n$, where $n=\prod_{i=1}^k p_i$.
		Additionally this prime factors will be sorted
		decreasingly, assuring a unique decomposition.

		\begin{Verbatim}
			s/x*/|&|/; :a
			  s/|\(xxx*\)\(\1\1*\)|/|\1<x>\2|/
			  s/\(x*\)<\(x*\)>\1/\1<\2x>/
			  s/<\(x*\)>|/|\1|/
			ta
		\end{Verbatim}

		\begin{enumerate}
			\item On the first line we wrap the {\tt x}s between pipes just to
				make processing simple, it is not required though.

			\item Then for a block of length $n$ this is when we find a 
				divisor $d>1$ of $n$, the backreference is placed twice to ensure
				$d<n$.  Since the algorithm for matching regular expressions is
				greedy, \verb|\1| will match the largest divisor smaller than $n$,
				meaning that $n/d$ is the smallest prime that divides $n$.

				The block $\texttt{|x}^n\texttt{|}$ will be converted into
				$\texttt{|x}^d\texttt{<x>x}^{n-d}\texttt{|}$.

			\item The next substitution gradually substracts $d$ from the
				``number'' after the \verb|>| symbol, adding one before it.
				In other words we just divided $n$ over $d$ using repeated
				substraction.
				So after applying this rule as many times as possible, we obtain:
				$\texttt{|x}^d\texttt{<x}^{n/d}\texttt{>|}$.

			\item And the final rule, applicable only when the division has
				completed, is to convert it back to the ``pipe separated
				unary format'' so that we can proceed to find newer factors:
				$ \texttt{|x}^n\texttt{|} \mapsto \ldots
				\mapsto \texttt{|x}^d\texttt{|x}^{n/d}\texttt{|}$
		\end{enumerate}

	\item[Logarithm: $x^n\mapsto y^{\lfloor\log_2 n\rfloor}$:]

		The idea here is to divide by 2 repeatedly until {\tt xx} can no longer
		be found.  Note that we divide by 2 rounding down by substituting a
		single (final) {\tt x} in the pattern at line 2 to an empty string.
		There may be a remaining {\tt x} which we remove at the end.

		\begin{Verbatim}
			:a; /xx/ {
			  s/x\(x\|\)/\1/g
			  s/^/y/
			ba }
			s/x//
		\end{Verbatim}

	\item[Minimum: $\texttt{x}^n\texttt{|x}^m \mapsto
		\texttt{x}^{\min\{n,m\}}$:]

		% sedcode minimum:
		Calculating the minimum between $n$ and $m$ is performed by
		substituting them leaving only the longest common prefix.  Such a
		transformation is simply \verb!s/\(x*\)x*|\1x*/\1/!.

	\item\todo{Test and write about maximum}
	% (tested) Maximum: x^n|x^m -> x^{\max\{n,m\}}: s/\(x*\)|\(\1\(x*\)\|x*\)/\1\3/

	\item[Fibonacci: $\texttt{x}^n\mapsto \texttt{x}^{\fib n}$:]

		Fibonacci is usually defined recursively by $\fib n = \fib (n-1) +
		\fib (n-2)$ with the base case $\fib n = n$ when $n<2$.  So in the
		implementation presented here we first handle the base case first by
		not adding pipes (which disables all substitutions on the following
		lines).

		The recursive case is handling by having our pattern space with the
		format $\texttt{x}^{n-i}\texttt{|x}^{\fib i-1}\texttt{|x}^{\fib i}$,
		and applying the inductive substitution $(a,b)\mapsto (b,a+b)$
		which increases by 1 the value at $i$:
		$\texttt{x}^{n-i}\texttt{|x}^{\fib i-1}\texttt{|x}^{\fib i}$
		$\mapsto$
		$\texttt{x}^{n-i-1}\texttt{|x}^{\fib i}\texttt{|x}^{\fib i+1}$.

		When $i$ reaches $n$, that is when there are no more {\tt x}s before
		the first pipe, we will have $\texttt{|x}^{\fib n-1}\texttt{|x}^{\fib n}$,
		so at that point we just leave the text after the last pipe:

		% sedcode fibonacci:
		\begin{Verbatim}
			/xx/ s/x$/||x/
			:a; s/x|\(x*\)|\(x*\)/|\2|\1\2/; ta
			s/.*|//
		\end{Verbatim}

\end{description}

%%%%%%%%%%%%%%%%%%%%%%%%%%%%%%%%%%%%%%%%%%%%%%%%%%%%%%%%%%%%%%%%%%%%%%%%%%%%%
\subsection{Binary Math}

\begin{description}
	\item[XOR: $a_n\ldots a_1 a_0\texttt{|}b_m\ldots b_1 b_0 \mapsto
		c_{\max\{n, m\}}\ldots c_1 c_0$ where $(c_i)=(a_i)\oplus(b_i)$:]$\,$

		Bit-wise XOR between $(a_i)$ and $(b_i)$, assuming as many leading zeros
		as needed.

		% sedcode bitwise-xor:
		\begin{Verbatim}
			s/$/|/
			:a; s/^|\(.*\)|/\1/; s/||//
			  s/0|\(.*\)\(.\)|/|\1|\2/
			  s/1|\(.*\)0|/|\1|1/
			  s/1|\(.*\)1|/|\1|0/; ta
			s/^0*//; s/^$/0/
		\end{Verbatim}

		\begin{enumerate}
			\item We start by adding a pipe to the end.
				We will store the result after that pipe.
			\item At the start of each iteration denoted by the label \verb|:a| we
				detect the cases where there are no more binary digits left in any
				of the operands.

				In the following 3 lines we will be processing
				one bit at a time starting from the right (least significant bit).
			\item The rule that we apply here $0\oplus x = x$.  We just maintain the
				least significant bit of the second number.
			\item Then we consider $1\oplus 0=1$.
			\item And the other case $1\oplus 1=0$.
			\item Finally we just remove leading zeroes and place a zero in place of
				an empty string.
		\end{enumerate}
\end{description}

\todo{Write about binary addition.}

\todo{Write about truncation.}

%%%%%%%%%%%%%%%%%%%%%%%%%%%%%%%%%%%%%%%%%%%%%%%%%%%%%%%%%%%%%%%%%%%%%%%%%%%%%
\subsection{Lookup Tables}

Many of the most complex programs written in \sed use the power of
backreferences in order to do translations which otherwise could be very
complex, here we will focus on two very practical use cases, converting from
decimal numbers to unary and viceversa.

\subsubsection{Decimal to Unary}

Let's say we want to convert the digits of the number {\tt 42} to unary:
{\tt |xxxx|xx}, we can use the following short program composed of just three
lines:

\begin{Verbatim}
	s/./|0123456789&!/g
	s/\(.\)[0-9]*\1//g
	:a; s/|!/|/g; s/.!/!x/g; ta
\end{Verbatim}

\begin{enumerate}
	\item with the first substitution we would obtain the seemingly useless
		string: {\tt |01234567894!|01234567892!}, yet there is an interesting
		pattern going on.
	\item If for each block we were to remove the duplicated numbers and
		everything in between, we would obtain {\tt |0123!|01!}, and that's
		what the second line does.

		Note that the number of digits left matches the corresponding digit
		that we had before.  That's because we inserted all the digits in order.

	\item Finally to obtain the {\tt x}s, we can iterate the exclamation marks
		to the left substituting the digits by {\tt x}s until they reach the
		pipes (when we remove them).
\end{enumerate}

\subsubsection{Unary to Decimal}
\todo{Copiar desde la cuadernola la explicaci\'on de las reverse lookup tables.}

\subsubsection{Generic Substitutions}
\todo{Escribir c\'omo se podr\'{\i}a usar una lookup table en general para
hacer una substitución arbitraria.}
% the example usually seen for "sed lookup tables" is that of substituting
% numbers via 0zero1one2two3three4four5five6six7seven8eight9nine, there has
% to be another simple example.
%   Maybe converting from hexa to binary requiring only two commands.

%%%%%%%%%%%%%%%%%%%%%%%%%%%%%%%%%%%%%%%%%%%%%%%%%%%%%%%%%%%%%%%%%%%%%%%%%%%%%
\subsection{Debugging}

When developing programs in \sed, more often than not we end up with mountains
of completely unreadable line noise which at a certain point, sometimes even
way before to be completed, leaves us in a state of absolute confusion.  Is at
this point that we really need to understand what's going on.

One of the simplest ways to try to understand the code, is to print the content
of the pattern space, in the most simple form we just place the {\tt p} command
whenever we want to, however if we want to print it at several places and
specially if there are newlines inside the pattern space, we might want to use
something like this instead:

\begin{Verbatim}
	... lots of code ...
	s/.*/At xyz: <<&>>/p;  s/[^<]*<<//;  s/>>$//
	... lots of code ...
\end{Verbatim}

Note that by using the previous code snippet we ``set to true'' the state of
whether a substitution was applied at that point.  If we don't care about its
state then we might just use that code, if not we have to maintain it by
basically duplicating our code around an ``if then else'':

\begin{Verbatim}
	... lots of code ...
	tTxyz; s/.*/At xyz (F): <<&>>/p; s/[^<]*<<//; s/>>$//; tFxyz
	:Txyz; s/.*/At xyz (T): <<&>>/p; s/[^<]*<<//; s/>>$//; :Fxyz
	... lots of code ...
\end{Verbatim}

This works because if the flag was true, it will jump to the {\tt (T)} part,
with these new substitutions restoring the flag to true.
Oterwise the conditional jump to {\tt Fxyz} will reset it to false
before continuing.

If the hold space has to be printed as well, then one simple solution is
logging the pattern space, followed by the {\tt x} command, that logging
the hold space and finally switching back the spaces with the {\tt x}
command once again.

\todo{add more tools}

%%%%%%%%%%%%%%%%%%%%%%%%%%%%%%%%%%%%%%%%%%%%%%%%%%%%%%%%%%%%%%%%%%%%%%%%%%%%%
\subsection{Jump Tables}

\todo{complete jump tables}
